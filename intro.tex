\section{Introduction}


\subsection{Measuring Storage Resource Consumption}

% storeage resource intro
Measuring consumption of storage resources is distintively different from
measuing consumption of batch jobs. Where jobs typically have exclusive access
to a single resource (a cpu core) for a given amount of time, a storage
resource\footnote{The authors are well aware that storage systems are created
by pooling togther disks, but this is transparent for the user, and typically
not used when accounting for resource consumption.} is almost always shared
among several parties, with each of these participants use a part of the
resource. Futhermore the usage of each participant will vary over time, and as
data tend to be long-lived there is no fixed start and end time of the 

% integrals and spikes
To exactly measure the amount of consumed resource\ldots




\subsection{Relation to the Usage Record format}

The record format described in this document, is clearly related to the usage
record format recommendation of the OGF 98 standard, is it tries to achive a
shared record format for accounting consumed resources. Furthermore it shares
several element names and semantics of the fields.

Wheather to define a completely new format, or partly reuse the usage record
format has not been decided.


\subsection{Record Structure}

The structure of the format described in this document can be split into
logical parts, each describing an aspect of the resource consumption. The parts
are:

\begin{description}

\item[Resource] Fields describing the system the resource was consumed on. Can
specify a certain subsystem of the storage system.

\item[Consumption Details] Fields describing what is consuming the data. E.g.,
storage classes, number of files, directory path, etc.

\item[Identity] Fields describing the person/group which is accountable for the
resource consumption.

\item[Resource Consumption] Fields describing how much of the described
resource has been used.

\end{description}

Please note that this organization, is not directly reflected in record format.
It is merely a good mental model to have in mind.


