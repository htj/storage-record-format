
\section{Record Properties}

This section describes the record properties and their fields and attributes.

The format of the record is XML, using QNames. There is currently no name space
defined. The examples use an ``ns'' name space as placeholder. All time and
duration formats are ISO types~\cite{iso8601}. These design choices are made in
order to keep the format as close as possible to OGF usage record format.

Many of the properties presented in this section are optional, however a few
are not. For the required properties, it is explicitly listed, that they must
be present in the record. None of the properties are allowed to be repeated.

A record should only represent a single identity. This identity can either be a
person of a group of users. If a record contains both user and group
information, the implementation should assume that the resources have been
consumed by the user in the context of the group information.


%each field should have:
% name
% precise semantic description
% should be included or not
% alternative/previous/discussed names
% examples

% record metadata

\subsection{RecordIdentity}

This property describes the identity of the record. The field has two
attributes: recordId and createTime. The recordId should be constructed in such
a way that it is globally unique and records with the same value is not
generated accidentally. Hence this field can be used to identify the record,
and be used duplicate detection. The createTime attribute describes when the
record was created.

The field is similar to the field with the same name in the Usage Record
standard.

\begin{itemize}
\item The RecordIdentity property MUST be present in the record.
\item The RecordIdentity field MUST NOT have a value.
\item The recordId attribute MUST be present in the record.
\item The recordId attribute MUST have the type string.
\item The createTime attribute MUST be present in the record.
\item The createTime attribute MUST be an ISO timestamp.
\end{itemize}

{\bf Example}
\begin{verbatim}
<ns:RecordIdentity
ns:createTime="2010-11-09T09:06:52Z"
ns:recordId="host.example.org/sr/87912469269276"/>
\end{verbatim}

% where block

\subsection{StorageSystem}

This property describes the storage system on which the resources has been
consumed. This value should be chosen in such a way that it globally identifies
the storage system, on which resources are being consumed. E.g., the FQDN of
the storage system could be used.

In grid lingo, this would be a storage element.

\begin{itemize}
\item The StorageSystem property SHOULD be present in the record.
\item The StorageSystem field MUST have the type string.
\item The StorageSystem value SHOULD be constructed in such a way, that it
    globally identifies the storage system.
\end{itemize}

{\bf Example}
\begin{verbatim}
<ns:StorageSystem>host.example.org</ns:StorageSystem>
\end{verbatim}


\subsection{StorageShare}

This property describes the part of of the storage system which is accounted
for in the record. For a storage system, which is split into several logical
parts, this can be used to account for consumption on each of these parts.
The value should be able to identity the share of the storage system, given
the storage system property.

This property has previously been named StoragePartition.

\begin{itemize}
\item The StorageShare field type MUST be a string.
\end{itemize}

{\bf Example}
\begin{verbatim}
<ns:StorageShare>pool-003</ns:StorageShare>
\end{verbatim}


\subsection{StorageMedia}

This property describes the media type of storage is accounted for in the
record, e.g., ``disk'' or ``tape''. This allows for accounting of different
backend storage types. This field should not be used if the data accounted for
in the record is on different media types.

The following values must be understood by the implementation at which records
are registered to: ``tape'', ``disk'', and ``solidstatedisk''.

This property has previously been named StorageType.

\begin{itemize}
\item The StorageType field type MUST be a string.
\end{itemize}

{\bf Example}
\begin{verbatim}
<ns:StorageType>disk</ns:StorageType>
\end{verbatim}


% what block

\subsection{StorageClass}

This property describes the class of the stored data, e.g., "pinned",
"replicated", "precious". This is a descriptive value, which allows the storage
system to provide details about the stored data.

\begin{itemize}
\item The StorageClass field type MUST be a string.
\end{itemize}

{\bf Example}
\begin{verbatim}
<ns:StorageClass>replicated</ns:StorageClass>
\end{verbatim}


\subsection{FileCount}

This property describes the number of files which are accounted for in the
record.

\begin{itemize}
\item The FileCount field type MUST be a positive non-zero integer.
\end{itemize}

{\bf Example}
\begin{verbatim}
<ns:FileCount>4</ns:FileCount>
\end{verbatim}



\subsection{DirectoryPath}

This property describes the directory path being accounted for. If the property
is included in the record, the record should account for all usage in the
directory and only that directory.


\begin{itemize}
\item The DirectoryPath field type MUST be a string.
\end{itemize}

{\bf Example}
\begin{verbatim}
<ns:DirectoryPath>/projectA</ns:DirectoryPath>
\end{verbatim}


% identity block

\subsection{SubjectIdentity}

This property is a container for all user and group properties. The property
itself should be able. The purpose of the property is to clearly mark a
property as describing a user or group, i.e., then entity accountable for the
resource consumption. The property is similar to the UserIdentity block in the
OGF Usage Record format, but it can also be used for describing group
affiliations.

\begin{itemize}
\item The SubjectIdentity property SHOULD be present in the record.
\item The SubjectIdentity property SHOULD include at least one subelement.
\item The SubjectIdentity field MUST NOT exist.
\end{itemize}

{\bf Example}
\begin{verbatim}
<ns:SubjectIdentity>
</ns:SubjectIdentity>
\end{verbatim}


\subsection{LocalUserName}

The local user name of the storage system accountable for the resource
consumption. Can be OS level or an internal user name in the storage system.

\begin{itemize}
\item If included, the LocalUserName property MUST be under the
SubjectIdentity.
\item The LocalUserName field type MUST be a string.
\end{itemize}

{\bf Example}
\begin{verbatim}
<ns:LocalUserName>johndoe</ns:LocalUserName>
\end{verbatim}


\subsection{LocalUserGroup}

The local user group of the storage system accountable for the resource
consumption. Can be OS level or an internal group in the storage system.

\begin{itemize}
\item If included, the LocalUserGroup property MUST be under the
SubjectIdentity.
\item The LocalGroupName field type MUST be a string.
\end{itemize}

{\bf Example}
\begin{verbatim}
<ns:LocalGroupName>binarydataproject</ns:LocalGroupName>
\end{verbatim}


\subsection{UserIdentity}

The global identity of the user accountable for the resource consumption. The
property should identity the user globally, such that clashes does not happen
accidentally, e.g., it could be an X509 identity.

\begin{itemize}
\item If included, the UserIdentity property MUST be under the
SubjectIdentity.
\item The UserIdentity field type MUST be a string.
\end{itemize}

{\bf Example}
\begin{verbatim}
<ns:UserIdentity>/O=Grid/OU=example.org/CN=John Doe</ns:UserIdentity>
\end{verbatim}


\subsection{Group}

The global group accountable for the resource consumption. The property should
identify the group globally, such that clashes does not happen accidentally,
e.g., using a FQDN to construct it. In grid terms, this would typically be the
VO name.

\begin{itemize}
\item If included, the Group property MUST be under the SubjectIdentity.
\item The Group field type MUST be a string.
\end{itemize}

{\bf Example}
\begin{verbatim}
<ns:Group>binarydataproject.example.org</ns:Group>
\end{verbatim}


\subsection{GroupPartition}

The part of a group accountable for the resource consumption. This is useful in
some projects which has several sub-groups which are each accountable for their
resource consumption. In grid terms this would typically be a group within a
VO.

\begin{itemize}
\item If included, the GroupPartition property MUST be under the
SubjectIdentity.
\item The GroupPartition field type MUST be a string.
\item The Group property MUST exist in the record of GroupPartition is
    specified.
\end{itemize}

{\bf Example}
\begin{verbatim}
<ns:GroupPartition>ukusers</ns:GroupPartition>
\end{verbatim}


\subsection{GroupRole}

The role of the group accountable for resource consumption. Roles are sometimes
used to denote special tasks within a group or similar. In grid terms, this
would be the role in a VO.

\begin{itemize}
\item If included, the GroupRole property MUST be under the SubjectIdentity.
\item The GroupRole field type MUST be a string.
\item The Group property MUST exist in the record of GroupRole is specified.
\end{itemize}

{\bf Example}
\begin{verbatim}
<ns:GroupRole>analysis</ns:GroupRole>
\end{verbatim}


\subsection{GroupAuthority}

The authority which the group is created under or controls the group. This can
be useful, if there are group name clashes, or to see the group authority is in
certain security infrastructures. In grid terms, this would be CA issuing the
certificate for the group.

\begin{itemize}
\item If included, the GroupAuthority property MUST be under the
SubjectIdentity.
\item The GroupAuthority field type MUST be a string.
\item The GroupAuthority property MUST exist in the record of GroupPartition
    is specified.
\end{itemize}

{\bf Example}
\begin{verbatim}
<ns:GroupAuthority>/O=Grid/OU=example.org/CN=host/auth.example.org
</ns:GroupAuthority>
\end{verbatim}


% consumption block

\subsection{MeasureTime}

A timestamp indicating when the measurement of the resource consumption was
made.

\begin{itemize}
\item The MeasureTime field SHOULD be present in the record.
\item The MeasureTime field type MUST be an ISO timestamp.
\end{itemize}

{\bf Example}
\begin{verbatim}
<ns:MeasureTime>2010-10-11T09:31:40Z</ns:MeasureTime>
\end{verbatim}


\subsection{ValidDuration}

A duration indicating for how long time the measurement is valid from its
measurement time. The record must only be used accounting for the time frame in
which it is valid. The generation of records should be performed in such a way,
that gaps should not occur. Note that a record can be ``nullified'' if a newer
record is manifested, i.e., the most recent information should be used.

\begin{itemize}
\item The ValidDuration field SHOULD be present in the record.
\item The ValidDuration field type MUST be an ISO duration.
\end{itemize}

{\bf Example}
\begin{verbatim}
<ns:ValidDuration>PT3600S</ns:ValidDuration>
\end{verbatim}


\subsection{ResourceCapacityUsed}

An integer denoting the number of bytes used on the storage system. This is the
main metric for measuring resource consumption. It should include all resources
for which the user or project is accountable for. This can include reserved
space, file metadata, space used for redundancy in RAID setups, tape holes,
or similar. The decision about including such, is a decision left to the
resource owner the usage policy.

This property has previously been known as SpaceAllocated and SpaceUsed.

\begin{itemize}
\item The ResourceCapacityUsed property SHOULD be present in the record.
\item The ResourceCapacityUsed field type MUST be a positive integer.
\end{itemize}

{\bf Example}
\begin{verbatim}
<ns:ResourceCapacityUsed>14728</ns:ResourceCapacityUsed>
\end{verbatim}

\subsubsection*{Implementation Note}

Using bytes saves us from the argument of discussing if 1000 or 1024 should be
used as a base. However, this also means that the number reported can be very
high. Therefore any implementation should use at least a 64-bit integer to hold
this variable. However it is highly recommend to use a data type with no upper
limits (a signed 64-bit integer will overflow at 8 Petabyte).


\subsection{LogicalCapacityUsed}

An integer denoting the number of ``logical'' bytes used on the storage system
by the identity in the record. By `logical'' is meant the sum of the files
stored, i.e., excluding reservation, and any underlying replicas of files.

\begin{itemize}
\item The LogicalCapacityUsed field type MUST be a positive integer.
\end{itemize}

{\bf Example}
\begin{verbatim}
<ns:LogicalCapacityUsed>13617</ns:LogicalCapacityUsed>
\end{verbatim}

\subsubsection*{Implementation Note}

Same as for ResourceCapacityUsed property.


\section{Intentionally Left Out Properties}

% Note: These are still under discussion.
% These should be removed in the final standard, but are in the draft for
% discussion purposes.
In the preparation phase yielding the current draft of the record a number of
properties were in discussion to be included. They are listed here for the sake 
of completeness.

\begin{description}

\item[Site] This property would identify the site, at which the storage system
is located, allowing easy grouping of several storage systems. However such a
grouping is often defined elsewhere and is not always static.

\item[FileNames] Providing a list of file names in the record would allow
per-file accounting, or allow certain files to be accounted for separately.
However, the available properties like path, storage system, group, and 
user provide a sufficient definition of the resource consumption.
% However the number of files stored can be very large, and file names 
% occasionally contain sensitive information.

\item[File Metadata] The focus of the record is accounting. Any per-file data
or metadata are out of scope of the record.

\item[SpaceAvailable] This property would describe how much space is available
for the identity on the storage system. This property would however not report
any form of consumption and is often difficult to determine.

\item[Transfer Information] The original suggestion included properties for
describing how much data has been transferred. There are however a range of
issues with this: A. network resources are not storage resources. B: Users
transferring data are not necessarily the user owning the data. Thus, the  
accounting of the network usage should be defined elsewhere.

\item[StartTime / EndTime] They have been replaced by measurement time and 
valid duration as it is not really known when the actual measurement of the 
resource consumption is taken.

\end{description}


