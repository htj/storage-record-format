
\section{Record Properties}

This section describes the properties (fields) used in record.

%each field should have:
% name
% precise semantic description
% should be included or not
% alternative/previous/discussed names
% examples

\subsection{RecordIdentity}

This property describe the identity of the record. The field has to attributes:
recordId and createTime. The recordId attribute of the type string and should
be constructed in such a way that it is globally unique and records with the
same value is not generated accidently. Hence this field can be used to identiy
the record, and provide a basis for duplicate detection. The createTime
attribute is an ISO timestamp describing when the record was created.

The field is similar to the field with the same name in the Usage Record
standard.

\begin{itemize}
\item The RecordIdentity property MUST be present in the record.
\item The RecordIdentity field MUST NOT have any value.
\item The recordId attribute MUST be present in the record.
\item The recordID attribute MUST have the type string.
\item The createTime attribute MUST be present in the record.
\item The createTime attribute MUST be an ISO timestamp.
\end{itemize}

{\bf Example}
\begin{verbatim}
<ns:RecordIdentity
ns:createTime="2010-11-09T09:06:52Z"
ns:recordId="host.example.org/sr/87912469269276"/>
\end{verbatim}


\subsection{StorageSystem}

This property describes the storage system on which the resources has been
consumed. This value should be choosen in such a way that it globally
identifies the storage system, on which resource is being consumed. E.g., the
FQDN of the storage system could be used.

In grid lingo, this would be a storage element.

\begin{itemize}
\item The StorageSystem property SHOULD be present in the record.
\item The StorageSystem field MUST have the type string.
\item The StorageSystem value SHOULD be constructed in such a way, that it
    globally identifies the storage system.
\end{itemize}

{\bf Example}
\begin{verbatim}
<ns:StorageSystem>host.example.org</ns:StorageSystem>
\end{verbatim}


\subsection{StorageShare}

This property describes the part of of the storage system which is accounted
for in the record. For a storage system, which is split into several logical
parts, this can be used to account for consumption on each of these parts.
The value should be able to identity the share of the storage system, given
the storage system property.

This property has previously been named StoragePartition.

\begin{itemize}
\item The StorageShare field type MUST be a string.
\end{itemize}

{\bf Example}
\begin{verbatim}
<ns:StorageShare>pool-003</ns:StorageShare>
\end{verbatim}


\subsection{StorageType}

This property describes which type of storage is accounted for in the record,
e.g., ''disk'' or tape. This allows for explicit accounting for different types for
backend store.

\begin{itemize}
\item The StorageType field type MUST be a string.
\end{itemize}

{\bf Example}
\begin{verbatim}
<ns:StorageType>disk</ns:StorageType>
\end{verbatim}


\subsection{StorageClass}

This property describes the class of the stored data, e.g., "pinned",
"replicated", "precious". This is a descriptive value, which allows the storage
system to provide details about the stored data.

\begin{itemize}
\item The StorageClass field type MUST be a string.
\item The following values MUST be 
\end{itemize}


 - what
StorageClass
FileCount
DirectoryPath
FileNames




MeasureTime     - timestamp when the usage consumption was measured
  ValidDuration / ExpireTime
  StartTime, EndTime


SpaceUsed
SpaceReserved
SpaceAvailable

 - who/identity
   -- global
   GlobalUserIdentity
   Group
   GroupAuthority
   SubGroup
   GroupRole (name could be better

UserIdentity
VOInformation       - group / project
LocalUserName,
LocalUserGroup



